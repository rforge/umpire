% NOTE -- ONLY EDIT THE .Rnw FILE!!!  The .tex file is
% likely to be overwritten.
%
%\VignetteIndexEntry{Umpire Primer}
%\VignetteKeywords{Simulation, Microarray}
%\VignetteDepends{methods, stats, utils, graphics, grDevices}
%\VignettePackage{Umpire}

\documentclass[11pt]{article}
\usepackage{graphicx}
\usepackage{cite}
\usepackage{hyperref}
\usepackage{color}
\usepackage{multirow}
\pagestyle{myheadings}
\markright{Umpire}

\setlength{\topmargin}{0in}
\setlength{\textheight}{8in}
\setlength{\textwidth}{6.5in}
\setlength{\oddsidemargin}{0in}
\setlength{\evensidemargin}{0in}

\def\rcode#1{\texttt{#1}}
\def\rpkg#1{\texttt{#1}}
\def\fref#1{\textbf{Figure~\ref{#1}}}
\def\tref#1{\textbf{Table~\ref{#1}}}
\def\sref#1{\textbf{Section~\ref{#1}}}

\title{Umpire: The Ultimate Microarray Prediction, Inference, and
  Reality Engine}
\author{Jiexin Zhang and Kevin R. Coombes}
\date{14 July 2009}

\usepackage{Sweave}
\begin{document}


\maketitle
%\tableofcontents
%\listoffigures

\section{Introduction}

Version 1.0 of the Ultimate Microarray Prediction, Inference, and
Reality Engine (Umpire) is an R package that allows researchers to
simulate complex, realistic microarray data.  Statisticians and
bioinformaticians who develop and improve methods to analze microarray
data recognize that it is difficult to evaluate methods on real data
where ``ground truth'' is unknown, and they frequently turn to
simulations where they can control the true underlying structure.  In
many instances, however, the simulations that have been performed are
rather simplistic.  Often, genes are treated as independent, in spite
of the elaborate correlation structures that give rise to networks and
pathways in real biology.  Differential epxression is frequently
simulated using two homogeneous groups following nearly perfect normal
distributions, with the amount of differential expression identical
for all genes.  The \rpkg{Umpire} package, which is invoked by the command
\begin{Schunk}
\begin{Sinput}
> library(Umpire)
\end{Sinput}
\end{Schunk}
provides tools that allow users to simulate microarray data from a
more realistic model.

\section{The gene expression model}
\subsection{Engines}

The fundamental object in \rpkg{Umpire} is a ``random-vector
generator'' (RVG), which is represented by the \rcode{Engine} class.
Equivalently, each \rcode{Engine} object represents a specific
multivariate distribution, from which random vectors can be generated
using the generic \rcode{rand} method.  In Version 1.0 of
\rpkg{Umpire}, we include three basic building blocks for these kinds
of distributions: independent normal, independent log normal, and
multivariate normal.  The following example creates an object that
will generate vectors of length $3$.
\begin{Schunk}
\begin{Sinput}
> nGenes <- 3
> means <- rnorm(nGenes, 6, 1)
> sds <- 1/rgamma(nGenes, rate = 14, shape = 6)
> indn <- IndependentNormal(means, sds)
> summary(indn)
\end{Sinput}
\begin{Soutput}
An IndependentNormal object, representing a vector
of length 3 of independent normal random variables.
\end{Soutput}
\begin{Sinput}
> indn
\end{Sinput}
\begin{Soutput}
An object of class "IndependentNormal"
Slot "mu":
[1] 4.817261 5.168326 4.630687

Slot "sigma":
[1] 3.839131 2.109463 1.206702
\end{Soutput}
\end{Schunk}
Now we generate five vectors from this distribution.
\begin{Schunk}
\begin{Sinput}
> x <- rand(indn, 5)
> x
\end{Sinput}
\begin{Soutput}
         [,1]      [,2]      [,3]     [,4]     [,5]
[1,] 4.508216 11.083197 0.8589291 1.657415 9.542727
[2,] 4.330626  6.984256 5.8524106 4.963475 7.636607
[3,] 4.872833  3.611485 4.2672394 3.785379 1.902016
\end{Soutput}
\end{Schunk}

We use a similar method to create an object that generates independent
log normal data.
\begin{Schunk}
\begin{Sinput}
> nGenes <- 4
> logmu <- rnorm(nGenes, 6, 1)
> logsigma <- 1/rgamma(nGenes, rate = 14, shape = 6)
> indLN <- IndependentLogNormal(logmu, logsigma)
> indLN
\end{Sinput}
\begin{Soutput}
An object of class "IndependentLogNormal"
Slot "logmu":
[1] 4.666941 6.237990 5.137024 6.186839

Slot "logsigma":
[1] 2.144192 1.732211 2.085422 3.246242
\end{Soutput}
\end{Schunk}

In order to create a multivariate normal RVG, we must specify the mean
vector and the covariance matrix.  Here we start with the correlation
matrix for a simple two-dimensional RVG.
\begin{Schunk}
\begin{Sinput}
> a <- runif(1)
> b <- sqrt(1 - a^2)
> X <- matrix(c(a, b, -b, a), 2, 2)
\end{Sinput}
\end{Schunk}
Next, we choose random positive squared-eigenvalues.
\begin{Schunk}
\begin{Sinput}
> Lambda2 <- diag(rev(sort(rexp(2))), 2)
\end{Sinput}
\end{Schunk}
We combine these into a covariance matrix.
\begin{Schunk}
\begin{Sinput}
> Y <- t(X) %*% Lambda2 %*% X
\end{Sinput}
\end{Schunk}
Finally, we use the MVN constructor
\begin{Schunk}
\begin{Sinput}
> mvn <- MVN(c(0, 0), Y)
\end{Sinput}
\end{Schunk}
and use it to generate five random vectors.
\begin{Schunk}
\begin{Sinput}
> x <- rand(mvn, 5)
\end{Sinput}
\end{Schunk}

A general \rcode{Engine} is a list of RVG components, like those just
created from the \rcode{IndependentNormal} or \rcode{MVN}
constructors.  For example, we can create an RVG Engine with two
components using the command:
\begin{Schunk}
\begin{Sinput}
> engine <- Engine(list(indn, mvn))
> summary(engine)
\end{Sinput}
\begin{Soutput}
An Engine with 2 components.
\end{Soutput}
\end{Schunk}
\begin{Schunk}
\begin{Sinput}
> data <- rand(engine, 5)
> data
\end{Sinput}
\begin{Soutput}
           [,1]       [,2]      [,3]       [,4]       [,5]
[1,] -2.9760465 -1.4301555  6.672100  6.8468020  6.1459152
[2,]  7.1909455  3.4706992  4.673207  7.3232419  2.2245688
[3,]  3.2033364  4.1869240  3.738508  5.6753917  5.1522911
[4,] -0.7254035 -0.6867143 -0.325851  1.7236900 -0.3701668
[5,] -0.9381583  1.3874326 -1.396581 -0.3979416 -1.2036734
\end{Soutput}
\end{Schunk}

\section{Additive and Multiplicative Noise}

We model the observed signal, $Y_{gi}$, for gene $g$ in sample $i$ as: 
\[Y_{gi} = S_{gi} * exp(H_{gi}) + E_{gi}\]
where
\[S_{gi} = \mbox{true biological signal}\]
\[H_{gi} = \mbox{multiplicative noise}\]
\[E_{gi} = \mbox{additive noise}\]

The noise model represents technical noise that is layered on top of
any biological variability when measuring gene expression in a set of
samples.  For example, background noise is usually additive to the
signal, and the variation between the signal pixels, which is
independent of the magnitude of the signal, is the representative
multiplicative noise~\cite{pmid11595791}.  We modeled both additive
and multiplicative noise as normal distribution:
\[E_{gi} \sim \mbox{Normal}(\nu,\tau)\]
\[H_{gi} \sim \mbox{Normal}(0,\phi)\]
Note that we allow the additive noise to include a bias term ($\nu$)
that may represent, for example, a low level of cross-hybridization
provding some level of signal at all genes.

The noise model is represented in the \rcode{Umpire} package by the
\rcode{NoiseModel} class.  You create a \rcode{NoiseModel} object by
supplying values for $\nu$, $\tau$, and~$\phi$.
\begin{Schunk}
\begin{Sinput}
> noise <- NoiseModel(30, 40, 0.1)
> noise
\end{Sinput}
\begin{Soutput}
An object of class "NoiseModel"
Slot "additiveOffset":
[1] 30

Slot "additiveScale":
[1] 40

Slot "multiplicativeScale":
[1] 0.1
\end{Soutput}
\end{Schunk}
Use the \rcode{blur} function to add noise to a data matrix. For
example, 
\begin{Schunk}
\begin{Sinput}
> ndata <- blur(noise, data)
> summary(data)
\end{Sinput}
\begin{Soutput}
       V1                V2                V3                V4         
 Min.   :-2.9760   Min.   :-1.4302   Min.   :-1.3966   Min.   :-0.3979  
 1st Qu.:-0.9382   1st Qu.:-0.6867   1st Qu.:-0.3259   1st Qu.: 1.7237  
 Median :-0.7254   Median : 1.3874   Median : 3.7385   Median : 5.6754  
 Mean   : 1.1509   Mean   : 1.3856   Mean   : 2.6723   Mean   : 4.2342  
 3rd Qu.: 3.2033   3rd Qu.: 3.4707   3rd Qu.: 4.6732   3rd Qu.: 6.8468  
 Max.   : 7.1909   Max.   : 4.1869   Max.   : 6.6721   Max.   : 7.3232  
       V5         
 Min.   :-1.2037  
 1st Qu.:-0.3702  
 Median : 2.2246  
 Mean   : 2.3898  
 3rd Qu.: 5.1523  
 Max.   : 6.1459  
\end{Soutput}
\begin{Sinput}
> summary(ndata)
\end{Sinput}
\begin{Soutput}
       V1                V2               V3               V4           
 Min.   :-16.697   Min.   :-17.55   Min.   :-4.748   Min.   :-18.66656  
 1st Qu.:  6.581   1st Qu.: 13.12   1st Qu.:14.369   1st Qu.:  0.04055  
 Median :  9.120   Median : 19.13   Median :24.952   Median : 19.29936  
 Mean   : 18.666   Mean   : 38.02   Mean   :19.611   Mean   : 23.22860  
 3rd Qu.: 31.278   3rd Qu.: 77.94   3rd Qu.:30.883   3rd Qu.: 30.70218  
 Max.   : 63.050   Max.   : 97.46   Max.   :32.597   Max.   : 84.76746  
       V5        
 Min.   :-34.57  
 1st Qu.: 15.27  
 Median : 40.75  
 Mean   : 30.99  
 3rd Qu.: 65.20  
 Max.   : 68.32  
\end{Soutput}
\end{Schunk}

\section{Gene Expression}

Let $T_{gi}$ denote the expression of a transcriptionally active gene
$g$ in sample $i$.  For most purposes, We allow $T_{gi}$ to follow a
log-normal distribution ($\log(T_g) \sim Normal(\mu_g,\sigma_g$).  In a
class of samples, the mean expression of gene $g$ on the log scale is
denoted by $\mu_g$ and the standard deviation on the log scale is
$\sigma_g$.  Both $\mu_g$ and $\sigma_g$ are properties of the gene
itself and the sample class.  Suppose, for example, in a class of
samples, that on average, $g_1$ expresses at a higher level than
$g_2$, and the variance of $g_1$ is smaller than $g_2$. Then,
$\mu_{g_1} > \mu_{g_2}$ and $\sigma_{g_1} < \sigma_{g_2}$.

Within a given simulation, we typically place hyperdistributions on
the log-normal parameters $\mu_g$ and $\sigma_g$.  We take $\mu_g \sim
Normal(\mu_o, sigma_0)$ to have a normal distribution with mean
$\mu_0$ and standard deviation $\sigma_0$. We take the $\sigma_g$ to
have an inverse gamma distribution with $rate$ and $shape$ parameters.
Reasonable values for the hyperparameters can be estimated from real
data.  For instance, $\mu_0=6$ and $\sigma_0=1.5$ are typical values
on the log scale of a microarray experiment using Affymetrix HG-U133A
chip.  The parameters for the inverse gamma distribution are
determined by the method of moments from thwe desired mean and
standard deviation; we have found that a mean of $0.65$ and a standard
deviation of $0.01$ (for which rate$ = 28.11$ and shape$ = 44.25$)
produce reasonable data.

Thus, we can create a simulation engine of this type by
\begin{Schunk}
\begin{Sinput}
> nGenes <- 4000
> mu0 <- 6
> sigma0 <- 1.5
> rate <- 28.11
> shape <- 44.25
> logmu <- rnorm(nGenes, mu0, sigma0)
> logsigma <- 1/rgamma(nGenes, rate = rate, shape = shape)
> indLN <- IndependentLogNormal(logmu, logsigma)
> engine <- Engine(list(indLN))
\end{Sinput}
\end{Schunk}

\subsection{The Multi-hit Model of Cancer}

The multiple hit theory of cancer was first proposed by Carl Nordling
in 1953~\cite{nordling} and extended by Alfred Knudson in 1971.  The
basic idea is that cancer can only result after multiple insults
(mutations; hits) to the DNA of a cell.  

\subsection{Active and Inactive Genes}

We model the true biological signal $S_{gi}$, for gene $g$ in sample
$i$, as a  mixture:
\[S_{gi} \sim (1-z_g) *\delta_0 + z_g * T_{gi}\]
In this model, $\delta_0$ is a point mass at zero, $z_g$ defines the
activity state (1 = active, 0 = inactive), and $T_{gi}$ is the
expression of a transcriptionally active gene.By allowing for some
genes to be transcriptionally inactive, this design takes into account
that the transcriptional activity of most genes is conditional on the
biological context.  For example, tissue-specific genes, unlike
housekeeping genes, only express in certain tissue samples.  Activity
is modeled in \rcode{Umpire} using a binomial distribution, $z_g \sim
Binom(p_0)$.  To create a simulation engine that inciorproates
transcriptional activation, we write
\begin{Schunk}
\begin{Sinput}
> p0 <- 0.8
> engine <- EngineWithActivity(p0, list(indLN))
> summary(engine)
\end{Sinput}
\begin{Soutput}
An Engine with 1 components.
Fraction of active genes 1
\end{Soutput}
\end{Schunk}





\subsection{ Correlated blocks of genes}
  
Instead of simulating genes as independent entities, we consider
blocks of correlated genes. Biologically, genes are usually
interconnected in networks and pathways. Often, clustering methods are
used to group genes into correlated blocks. Thus, it is natural to
simulate microarray experiments from the perspective of blocks. Since
the size of the blocks and degree of correlations among genes within a
block depend on biological condition of samples, they need to be
simulated within a reasonable range in order to study their effect on
the microarray data analysis. As shown in Table~\ref{tab:param}, we
allow the mean block size, \textit{bs}, to range from 1 to 1000, and
the sizes of gene blocks to vary around the pre-defined mean block
size. To be more specific, the block size follow normal distribution
with mean \textit{bs} and standard deviation $0.3*bs$. $bs=1$ is a
special case in which standard deviation of block size = 0. Thus, when
$bs=1$, there are no correlated blocks, which means all genes are
independent of each other. On the other extreme, $bs=1000$
demonstrates the situation in which the common theme is large networks
involving many genes. We have also simulated blocks where the standard
deviation = 0 for all mean block size, under which circumstance all
blocks in a microarray experiments have the exactly same
sizes. Comparing with variable block size, the setting of constant
block size affects the variability of the parameters of
interest. However, because we believe that the variable block size is
more realistic, we will present in this paper only the results
obtained from variable block size. Comparison between results from
constant block size and from variable block size is shown in
supplementary material.

As discussed in previous paragraph, we need to simulate the
correlation among genes within a block. The correlation matrix for a
block $b$, $cor.matrix_b$, has 1 on the diagnal and $\rho_b$ or
-$\rho_b$ in off-diagnal places. We allow $\rho$ to follow a beta
distribution with parameters $p$ and $w$: $Beta(pw,(1-p)*w)$. With the
setting of $p$ = 0.6 and $w$ = 5, most blocks are relatively highly
correlated (mean of $\rho$ is around 0.6). The portion of negatively
correlated genes within a block is denoted by parameter
\textit{p.neg}. In the simplest set-up, we have all genes in the same
block to have the same correlation $\rho_b$. Because $\rho_b$ is
always positive, this set-up means there is not negatively correlated
genes within a block (\textit{p.neg} = 0). In more complicated set-up,
we allow the portion of negatively correlated genes within a block
(\textit{p.neg}) to be supported on [0,0.5]. Thus, we have
\textit{p.neg} = 0.5-abs(x-0.5) where x follow some beta
distribution. Three scenarios were considered: (1) x~beta(1,1), in
which case the \textit{p.neg} is uniformly distributed between 0 and
0.5; (2) x~beta(5,5), in which case it is more likely that the
\textit{p.neg} is close to 0.5; (3) x~beta(0.5,0.5), in which case it
is more likely that the \textit{p.neg} is close to
0. Figure~\ref{fig:param1}c shows the histogram of pair-wise
correlations within 10000 genes and mean block size 50 when
\textit{p.neg} is uniformly distributed between 0 and 0.5. The
distributions of pair-wise correlations for all four \textit{p.neg}
scenarios is shown in supplementary materials (fig:pneg)

We allow the log expression values of genes in a block to follow a
multi-variate normal (MVN) distribution. The mean for the MVN object
is defined by $\mu_g$, and the covariance matrix is defined as the
following: 
\[cov.matrix[i,j] = cor.matrix[i,j] * \sigma_{g_i} * \sigma_{g_j}\]
where $\sigma_{g_i}$ defines the standard deviation of gene $i$, which
follows the inverse gamma distribution as described in previous
section. $cor.matrix$ denotes the correlation matrix as described in
previous paragraph.

In previous section, we mentioned that some genes would be
transcriptionally inactive under certain biological
conditions. Instead of simulating this active status for each gene, we
simulate the whole block of genes being transcriptionally active or
inactive. This follows the argument that the whole pathway/network
could be turned on or off under certain bioligical conditions. The
active status of blocks for the microarray experiment follows a
binomial distribution with parameter $\pi$ which defines the portion
of transcriptionally inactive blocks. When a block is turned off,
$z_g$, the status indicator, is set to be 0 for all genes in this
block, so that the true biological signals for these genes are zero. 


\subsection{ Normal \textit{vs} cancer samples }

We simulate normal samples being a homogeneous population with
\textit{nGenes} genes and \textit{nSamples} samples. We allow
\textit{nSamples} to vary from 10 to 100 in order to study the effect
of number of independent observations on various test statistics. The
same number of cancer samples are being generated with a portion of
differentially expressed genes. We simulate differentially expressed
genes in cancer samples by changing their mean expression
values. Instead of changing individual genes, we perform this mean
altering to blocks of genes in order to simulate the affect of cancer
pathology on certain pathway/networks. \textit{p.diff} is used to
define the percentage of differentially expressed blocks which are
then randomly selected from transcriptionally active blocks. We keep
transcriptionally inactive blocks inactive in both normal and cancer
samples in this setting. However, it is possible that an inactive
block of genes in normal samples being turned on in cancer samples, or
vise versa, when certain carcinogens work through pathways that are
supposed to be off, or on, in normal samples. We will incorporate this
level of complication in later implementations. The parameter
$diff.mean$ denotes the absolute changes of the mean expression values
on log scale for a block of genes. $diff.mean$ follows a gamma
distribution with parameter $\alpha$ and $\beta$
(Figure~\ref{fig:param1}d). The $\alpha$ and $\beta$ are both set be
10 so that the absolute fold change on log scale is 1 (thus 2 fold
change on raw scale), and the long tail on the right hand side of the
distribution indicates a few genes would have large fold changes. A
gene in the changed block is randomly assigned to be up-regulated or
down-regulated in cancer samples. 

Using parameters described above and summarized in
Table~\ref{tab:param}, Figure~\ref{fig:param1}e shows the distribution
of the average of log expression values of 10000 genes from 10
simulated normal samples given the mean block size being 100. The
bimodal profile is due to the fact that part of genes are
transcriptionally inactive. Similarly, 10 cancer samples were
generated with 10\% differentially expressed
blocks. Figure~\ref{fig:param1}g shows the log mean expression values
of these 10000 genes in normal samples verse those in cancer
samples. Red dots represent those genes that differentially expressed
in cancer samples. 

\setcounter{table}{0}
\begin{table}[h]
\begin{tabular}{rrrr}
  \hline
\textbf{group} & \textbf{parameter} & \textbf{value} & \textbf{description}\\ \hline
\multirow{2}{*}{log mean of true biological signal $\mu_{g}$} & $\mu_0$ & 6 & mean\\
& $\sigma_0$ & 1.5 & std\\ \hline
\multirow{2}{*}{log std of true biological signal $\sigma_{g}$} & \textit{sMean} & 0.65 & mean\\
& \textit{sVar} & 0.01 & variance\\ \hline
multiplicative noise $H_{gi}$ & $\phi$ & 0.1 & std\\ \hline
\multirow{2}{*}{additive noise $E_{gi}$} & $\nu$ & 30 & mean\\
& $\tau$ & 40 & std\\ \hline

\multirow{2}{*}{block correlation $\rho$} & $p$ & 0.6 & beta dist. parameter 1\\
& $w$ & 5 & beta dist. parameter 2\\ \hline

\multirow{2}{*}{diff.mean} & $\alpha$ & 10 & gamma dist. parameter 1\\
& $\beta$ & 10 & gamma dist. parameter 2\\ \hline

\multirow{5}{*}{constant} & $\pi$ & 0.3 & portion of\\
& & & transcriptionally active blocks\\
& \textit{bs} & 1,5,10,50,100, & number of\\
& & 250,500,1000 & genes per block size\\
& \textit{nGenes} & 10000 & number of\\
& & & genes in microarray experiment\\
& \textit{nSamples} & 10,25,50,100 & number of\\
& & & samples in each condition\\
& \textit{p.diff} & 0.1 & portion of\\
& & & differentially expressed genes\\ \hline
\end{tabular}
\caption{Parameters used in simulation}
\label{tab:param}
\end{table}

\section{Appendix}

This analysis was performed in the following directory:
\begin{Schunk}
\begin{Sinput}
> getwd()
\end{Sinput}
\begin{Soutput}
[1] "/home/jxzhang/z/Kevin/umpire/SandBox/Umpire.Rcheck/inst/doc"
\end{Soutput}
\end{Schunk}
This analysis was performed in the following software environment:
\begin{Schunk}
\begin{Sinput}
> sessionInfo()
\end{Sinput}
\begin{Soutput}
R version 2.10.1 (2009-12-14) 
i686-pc-linux-gnu 

locale:
 [1] LC_CTYPE=en_US       LC_NUMERIC=C         LC_TIME=en_US        LC_COLLATE=en_US    
 [5] LC_MONETARY=C        LC_MESSAGES=en_US    LC_PAPER=en_US       LC_NAME=C           
 [9] LC_ADDRESS=C         LC_TELEPHONE=C       LC_MEASUREMENT=en_US LC_IDENTIFICATION=C 

attached base packages:
[1] tools     stats     graphics  grDevices utils     datasets  methods   base     

other attached packages:
[1] Umpire_1.0.0
\end{Soutput}
\end{Schunk}

\end{document}

